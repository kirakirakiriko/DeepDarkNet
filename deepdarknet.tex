\documentclass[12pt]{beamer}
\usetheme{Berkeley}
\usepackage[utf8]{inputenc}
\usepackage[german]{babel}
\usepackage[T1]{fontenc}
\usepackage{amsmath}
\usepackage{amsfonts}
\usepackage{amssymb}
\usepackage{graphicx}
\author{Nicas Heydorn \and Kevin Kahrau \and Christofer Ostwald}
\title{Deepnet \& Darknet}
%\setbeamercovered{transparent} 
%\setbeamertemplate{navigation symbols}{} 
%\logo{} 
\institute{Nordakademie - Hochschule der Wirtschaft} 
%\date{} 
\subject{Technische Grundlagen 2} 
\begin{document}

\begin{frame}
\titlepage
\end{frame}

\begin{frame}
\tableofcontents
\end{frame}

\section{Definition}
\begin{frame}{Definition}
	Deep Web - \emph{Verstecktes Web}
	\begin{itemize}
		\item Teil des WWW welches nicht über Suchmaschiene auffindbar ist
		\item Gegensatz Visible Web oder Surface Web
\item Inhalte sind nicht frei zugänglich oder nicht durch Suchmaschienen indezierbar
	\end{itemize}
\end{frame}

\section{Arten}
\begin{frame}{Opaque Web}
	\begin{itemize}
		\item könnten technisch indexiert werden ist jedoch zu aufwendig für den erwarteten Nutzen
		\item Webcrawler berücksichten oft nur 5-6 Verzeichnisebene und finden somit keine tief verschachtelten Dokumente
		\item Nur teilweise indezierbare Dateiformate (z.B PDF)
	\end{itemize}
\end{frame}
\begin{frame}{Private Web}
	\begin{itemize}
		\item keine Indexierung durch Zugriffsbeschränkung des Webmasters (Intranet, Authentication(User/Passwort), IP-Whitelist, Meta-Tag noindex/nofollow/noimageindex, "Robots Exclusion Standard")
	\end{itemize}
\end{frame}
\begin{frame}{Proprietary Web}
	\begin{itemize}
		\item Zustimmung von Nutzungsbedingen erforderlich
		\item Passwort erforderlich
		\item z.B. webbasierte Fachdatenbanken
	\end{itemize}
\end{frame}
\begin{frame}{Invisible Web}
	
\end{frame}
\begin{frame}{Truly Invisible Web}
\end{frame}

\end{document}